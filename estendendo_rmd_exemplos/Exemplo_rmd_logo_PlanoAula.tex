\documentclass[]{article}
\usepackage{lmodern}
\usepackage{amssymb,amsmath}
\usepackage{ifxetex,ifluatex}
\usepackage{fixltx2e} % provides \textsubscript
\ifnum 0\ifxetex 1\fi\ifluatex 1\fi=0 % if pdftex
  \usepackage[T1]{fontenc}
  \usepackage[utf8]{inputenc}
  \usepackage{eurosym}
\else % if luatex or xelatex
  \ifxetex
    \usepackage{mathspec}
  \else
    \usepackage{fontspec}
  \fi
  \defaultfontfeatures{Ligatures=TeX,Scale=MatchLowercase}
  \newcommand{\euro}{€}
\fi
% use upquote if available, for straight quotes in verbatim environments
\IfFileExists{upquote.sty}{\usepackage{upquote}}{}
% use microtype if available
\IfFileExists{microtype.sty}{%
\usepackage{microtype}
\UseMicrotypeSet[protrusion]{basicmath} % disable protrusion for tt fonts
}{}
\usepackage[margin=1in]{geometry}
\usepackage{hyperref}
\hypersetup{unicode=true,
            pdftitle={Plano Aula 18 e 19},
            pdfauthor={Markus Stein},
            pdfborder={0 0 0},
            breaklinks=true}
\urlstyle{same}  % don't use monospace font for urls
\usepackage{graphicx,grffile}
\makeatletter
\def\maxwidth{\ifdim\Gin@nat@width>\linewidth\linewidth\else\Gin@nat@width\fi}
\def\maxheight{\ifdim\Gin@nat@height>\textheight\textheight\else\Gin@nat@height\fi}
\makeatother
% Scale images if necessary, so that they will not overflow the page
% margins by default, and it is still possible to overwrite the defaults
% using explicit options in \includegraphics[width, height, ...]{}
\setkeys{Gin}{width=\maxwidth,height=\maxheight,keepaspectratio}
\IfFileExists{parskip.sty}{%
\usepackage{parskip}
}{% else
\setlength{\parindent}{0pt}
\setlength{\parskip}{6pt plus 2pt minus 1pt}
}
\setlength{\emergencystretch}{3em}  % prevent overfull lines
\providecommand{\tightlist}{%
  \setlength{\itemsep}{0pt}\setlength{\parskip}{0pt}}
\setcounter{secnumdepth}{0}
% Redefines (sub)paragraphs to behave more like sections
\ifx\paragraph\undefined\else
\let\oldparagraph\paragraph
\renewcommand{\paragraph}[1]{\oldparagraph{#1}\mbox{}}
\fi
\ifx\subparagraph\undefined\else
\let\oldsubparagraph\subparagraph
\renewcommand{\subparagraph}[1]{\oldsubparagraph{#1}\mbox{}}
\fi

%%% Use protect on footnotes to avoid problems with footnotes in titles
\let\rmarkdownfootnote\footnote%
\def\footnote{\protect\rmarkdownfootnote}

%%% Change title format to be more compact
\usepackage{titling}

% Create subtitle command for use in maketitle
\newcommand{\subtitle}[1]{
  \posttitle{
    \begin{center}\large#1\end{center}
    }
}

\setlength{\droptitle}{-2em}

  \title{Plano Aula 18 e 19}
    \pretitle{\vspace{\droptitle}\centering\huge}
  \posttitle{\par}
    \author{Markus Stein}
    \preauthor{\centering\large\emph}
  \postauthor{\par}
      \predate{\centering\large\emph}
  \postdate{\par}
    \date{14 and 16 October 2019}

\usepackage{fancyhdr}

\begin{document}
\maketitle

\addtolength{\headheight}{1.0cm} \pagestyle{fancyplain}
\rhead{\includegraphics[height=1.5cm]{Logo-40-anos-estatistica.png}}
\lhead{\includegraphics[height=1.5cm]{logoIME60.jpg}}
\chead{UNIVERSIDADE FEDERAL DO RIO GRANDE DO SUL \\
INSTITUTO DE MATEMÁTICA E ESTATÍSTICA \\
DEPARTAMENTO DE ESTATÍSTICA \\
\vspace{0.3cm}
MAT02023 - INFERÊNCIA B - 2019/2
} \renewcommand{\headrulewidth}{0pt}

\section{JORNADA 60 ANOS IME/UFRGS}\label{jornada-60-anos-imeufrgs}

Tarefa não presencial.

\textbf{Exercício 1}: Seja \(X_1\) uma única observação obtida da
distribuição \(Beta(\theta,1)\):

\begin{enumerate}
\def\labelenumi{\alph{enumi})}
\tightlist
\item
  Mostre que \(X_1^{\theta}\) é uma quantidade pivotal.
\item
  Construa um intervalo com 95\% de confiança utilizando a quantidade
  pivotal \(X_1^{\theta}\).
\item
  Assuma \emph{a piori} \(\theta \sim Gama(\alpha, \beta)\), encontre um
  intervalo \(1 - \alpha\) de credibilidade para \(\theta\). Compare os
  intervalos.
\item
  Comente sobre as suposições para construirmos intervalos segundo as
  duas abordagens.
\item
  Teste de hipóteses frequentistas e bayesianos também podem ser
  construídos com base nos intervalos de confiança e de credibilidade,
  respectivamente. Gere uma amostra de tamanho \(n=1\) da distribuição
  \(Beta(1,5; 1)\) e teste a hipótese \(H_0 : \theta \leq 1\) contra
  \(H_1: \theta > 1\).
\item
  Calcule o valor \(p\) para os testes acima.
\end{enumerate}

\vspace{1cm}

\textbf{Definição 2 Valor \(p\)}: Um `valor \(p\)' é uma estatística
de teste, \(p(\boldsymbol{X})\), satisfazendo
\(0 \leq p(\boldsymbol{x}) \leq 1\) para todo ponto amostral
\(\boldsymbol{x} \in \mathcal{X}\) e
\[ P_\theta (p(\boldsymbol{X}) \leq \alpha) \leq \alpha, \] para todo
\(\theta \in \Theta_0\) e todo \(0 \leq \alpha \leq 1\).

\vspace{0.5cm}

Lembrando do `Plano Aula 11 e 12' vimos a definição abaixo.

\vspace{1cm}

\textbf{Definição Nível descritivo (amostral) ou valor \(p\)}:
\(\hat \alpha\) é o menor nível \(\alpha\) para o qual a hipótese
nula seria rejeitada.

\subsection{\texorpdfstring{Quiz sobre valor
\(p\).}{Quiz sobre valor p.}}\label{quiz-sobre-valor-p.}

\begin{enumerate}
\def\labelenumi{\arabic{enumi}.}
\tightlist
\item
  Qual o significado do valor \(p\) na prática? Como a ciência tem
  utilizado o valor \(p\) para criar suas teorias? Cite exemplos.
\item
  Porque o uso do valor \(p\) tem sido muito criticado mais
  recentemente?
\item
  Qual sua conclusão sobre o problema. Indique alternativas ao valor
  \(p\).
\end{enumerate}

\vspace{0.5cm}

\begin{center}\rule{0.5\linewidth}{\linethickness}\end{center}

\subsubsection{Leitura: Ler seções 8.3.4 do livro Casella e
Berger.}\label{leitura-ler-seaaes-8.3.4-do-livro-casella-e-berger.}

\subsubsection{\texorpdfstring{Leitura 2: Ler `Testes Bayesianos',
seção 6.6 do livro Bolfarine e
Sandoval.}{Leitura 2: Ler Testes Bayesianos, seção 6.6 do livro Bolfarine e Sandoval.}}\label{leitura-2-ler-testes-bayesianos-seaao-6.6-do-livro-bolfarine-e-sandoval.}

\begin{center}\rule{0.5\linewidth}{\linethickness}\end{center}

\subsubsection{\texorpdfstring{Discussão: Controvérsia sobre
Significância Estatística e Valor
\(p\)}{Discussão: Controvérsia sobre Significância Estatística e Valor p}}\label{discussao-controvarsia-sobre-significancia-estatastica-e-valor-p}

\begin{itemize}
\tightlist
\item
  \emph{Scientists rise up against statistical significance}
  (\url{https://www.nature.com/articles/d41586-019-00857-9})

  \begin{itemize}
  \tightlist
  \item
    Idéia de Fisher sobre verossimilhança e testes de significância?
    Já dizia isso. (livro do Fisher ``Statistical Methods and
    Scientific Inference'', pgs. 72 e 35)\\
  \item
    Significado de probabilidade (Fisher, pg. 32)
  \item
    \textbf{Verossimilhança} \emph{versus} \textbf{probabilidade}.
  \end{itemize}
\end{itemize}

\vspace{0.5cm}

\begin{itemize}
\tightlist
\item
  (Artigo recente 3 páginas) \emph{Why are p-Values Controversial?}

  \begin{itemize}
  \tightlist
  \item
    (\url{https://amstat.tandfonline.com/doi/full/10.1080/00031305.2016.1277161\#.XaSPy_dRduQ})
  \end{itemize}
\end{itemize}

\vspace{0.5cm}

\begin{itemize}
\tightlist
\item
  (Artigo bastante completo) \emph{Statistical tests, P values,
  confidence intervals, and power: a guide to misinterpretations.}

  \begin{itemize}
  \tightlist
  \item
    (\url{https://www.ncbi.nlm.nih.gov/pubmed/27209009})
  \end{itemize}
\end{itemize}

\vspace{0.5cm}

\begin{itemize}
\tightlist
\item
  (Artigo recente 3 páginas) \emph{Are confidence intervals better
  termed â\euro{}œuncertainty intervalsâ\euro{}?}

  \begin{itemize}
  \tightlist
  \item
    (\url{https://www.bmj.com/content/bmj/366/bmj.l5381.full.pdf})
  \end{itemize}
\end{itemize}


\end{document}
